\section{3 elemes modell a koncentrált tömeg figyelembe vételével, koncentrált tömegmátrixszal}

\subsection{Általános adatok}

A csomópontok, az elemek, illetve ezek kapcsolata a (b) feladatban
tárgyaltakkal megegyezik.

\subsection{A globális merevségi mátrix}

A globális merevségi mátrix alakja is megegyezik a (b) feladatrészben
ismertetettekkel, vagyis:
\begin{myframe}
  \begin{equation}
    \rmat K^\d = \rmat K^\b
    \text.
  \end{equation}
\end{myframe}

\subsection{A globális merevségi mátrix}

A gerenda-elemekhez tartozó elemi tömegmátrixokat ebben a feladatban más
módszerrel fogjuk meghatározni. Egy koncentrált tömegmátrix paraméteresen:
\begin{myframe}
  \begin{equation}
    \rmat M_i = \frac{\rho_i A_i L_i}{2}\begin{bmatrix}
      1 & 0                   & 0 & 0                   \\
      0 & \sfrac{{L_i}^2}{12} & 0 & 0                   \\
      0 & 0                   & 1 & 0                   \\
      0 & 0                   & 0 & \sfrac{{L_i}^2}{12} \\
    \end{bmatrix}
    \label{eq:Mk-base}
    \text.
  \end{equation}
\end{myframe}
Az elemi koncentrált tömegmátrixok a (\ref{eq:Mk-base})-es egyenlet alapján
numerikusan:
\begin{myframe}
  \begin{align}
    \siplaces{4}
    \sifix{}
    \rmat M_1^\d = \rmat M_2^\d = \left[
      \begin{array}{*{8}{X{2cm}}}
        \pyc{print_matrix(V["M_d"][0], 1e-6, 1)}
      \end{array}
      \right]
    \,\mathrm{SI} \text,
    \\
    \siplaces{4}
    \sifix{}
    \rmat M_3^\d = \left[
      \begin{array}{*{8}{X{2cm}}}
        \pyc{print_matrix(V["M_d"][2], 1e-6, 1)}
      \end{array}
      \right]
    \,\mathrm{SI} \text.
  \end{align}
\end{myframe}
A globális tömegmátrix az alábbi 2 mátrix összegeként adódik:
\begin{myframe}
  \begin{equation}
    \rmat M^\d = \rmat M^\d_\text{elemek} + \rmat M^\d_{m_0}
    \text,
  \end{equation}
\end{myframe}
ahol $\rmat M^\d_{m_0} = \rmat M^\c_{m_0}$, $\rmat M^\d_\text{elemek}$ pedig
a \ref{fig:K-construction-b}. ábra alapján:
\begin{myframe}
  \begin{equation}
    \siplaces{4}
    \sifix{}
    \rmat M^\d_\text{elemek} = \left[
      \scalebox{.667}{$
          \begin{array}{*{8}{X{1.9cm}}}
            \pyc{print_matrix(V["M_e"], 1e-6, 1)}
          \end{array}
        $}
      \right]
    \,\mathrm{SI}\text.
  \end{equation}
\end{myframe}
A globális tömegmátrix tehát:
\begin{myframe}
  \begin{equation}
    \siplaces{4}
    \sifix{}
    \rmat M^\d = \left[
      \scalebox{.667}{$
          \begin{array}{*{8}{X{2cm}}}
            \pyc{print_matrix(V["M"]["d"], 1e-6, 1)}
          \end{array}
        $}
      \right]
    \,\mathrm{SI}\text.
  \end{equation}
\end{myframe}

\subsection{A frekvencia-egyenlet}

A peremfeltételek megegyeznek a (\ref{eq:cond-3})-os egyenletben
szereplő feltételekkel. Ezek alapján a kondenzált mátrixok:
\begin{myframe}
  \begin{alignat}{9}
    \widehat{\rmat M}^\d & =
    \siplaces{4}
    \sifix{}
    \left[
      \scalebox{.833}{$
          \begin{array}{*{6}{X{2cm}}}
            \pyc{print_matrix(V["M_kond"]["d"], 1e-6, 1)}
          \end{array}
        $}
      \right]
    \,\mathrm{SI}\text,
    \\
    \widehat{\rmat K}^\d & =
    \siplaces{4}
    \sifix{}
    \left[
      \scalebox{.833}{$
          \begin{array}{*{6}{X{2cm}}}
            \pyc{print_matrix(V["K_kond"][3], 1, 1e-3)}
          \end{array}
        $}
      \right]
    \cdot 10^3 \,\mathrm{SI}\text.
  \end{alignat}
\end{myframe}

A kondenzált frekvencia-egyenlet:
\begin{myframe}
  \begin{equation}
    \det \left(
    \widehat{\rmat K}^\d - \alpha^2 \widehat{\rmat M}^\d
    \right) = 0
    \text.
  \end{equation}
\end{myframe}

A szerkezet első három saját-körfrekvenciája, és sajátfrekvenciája:
\begin{myframe}
  \begin{alignat}{9}
    \alpha_1^\d & = \pyc{prin_TeX(V["alpha"]["d"][0], "", "4")} \,/\, \mathrm{s}
    \qquad      & \rightarrow \qquad
    f_1^\d      & = \pyc{prin_TeX(V["f"]["d"][0], "Hz", "4")}
    \text,                                                                       \\
    \alpha_2^\d & = \pyc{prin_TeX(V["alpha"]["d"][1], "", "4")} \,/\, \mathrm{s}
    \qquad      & \rightarrow \qquad
    f_2^\d      & = \pyc{prin_TeX(V["f"]["d"][1], "Hz", "4")}
    \text,                                                                       \\
    \alpha_3^\d & = \pyc{prin_TeX(V["alpha"]["d"][2], "", "4")} \,/\, \mathrm{s}
    \qquad      & \rightarrow \qquad
    f_3^\d      & = \pyc{prin_TeX(V["f"]["d"][2], "Hz", "4")}
    \text.
  \end{alignat}
\end{myframe}
