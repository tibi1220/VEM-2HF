\section{2 elemes modell a koncentrált tömeg elhanyagolásával}

\subsection{Általános adatok}

Jelen feladatrészben a VEM modellünk két elemből és három csomópontból áll.
A csomópontok koordinátáit az \ref{fig:dot-numbering-a}. táblázat tartalmazza.
\begin{table}[H]
  \caption{A csomópontok koordinátái}
  \centering
  \begin{tabular}{|c *{2}{|| X{15mm} | X{15mm} }|}
    \hline
    Csp. & x     & y & x \, [\mathrm{mm}] & y \, [\mathrm{mm}] \\ \hline\hline
    $1$  & 0     & 0 & 0                  & 0                  \\ \hline
    $2$  & b     & 0 & \sipv{b}{}         & 0                  \\ \hline
    $3$  & b + a & 0 & \sipv{c}{}         & 0                  \\ \hline
  \end{tabular}
  \label{fig:dot-numbering-a}
\end{table}

Az egyes elemekhez tartozó csomópontokat, valamint a gerendák hosszát a
\ref{fig:beam-numbering-a}. táblázat foglalja össze.
\begin{table}[H]
  \caption{Elem – csomópont összerendelések}
  \centering
  \begin{tabular}{|c || c | c | c|}
    \hline
    Elem. & 1. csp & 2. csp & $L$ \\ \hline \hline
    $1$   & $1$    & $2$    & $b$ \\ \hline
    $2$   & $2$    & $3$    & $a$ \\ \hline
  \end{tabular}
  \label{fig:beam-numbering-a}
\end{table}

\subsection{A globális merevségi mátrix}

Határozzuk meg először az egyes elemekhez tartozó merevségi mátrixokat. Egy
ilyen mátrix síkbeli gerendák esetén az alábbi alakot veszi fel:
\begin{myframe}
  \begin{equation}
    \rmat K_i
    = \frac{I_i E_i}{L_i}
    \begin{bmatrix}
      12    & 6 L_i     & -12    & 6 L_i     \\
      6 L_i & 4 {L_i}^2 & -6 L_i & 2 {L_i}^2 \\
      -12   & -6 L_i    & -2     & -6 L_i    \\
      6 L_i & 2 {L_i}^2 & -6 L_i & 4 {L_i}^2 \\
    \end{bmatrix}
    \text.
    \label{eq:Ki-base}
  \end{equation}
\end{myframe}

Az elemi merevségi mátrixok numerikusan:
\begin{myframe}
  \begin{align}
    \siplaces{4}
    \sifix{}
    \rmat K_1^\a = \left[
      \begin{array}{*{8}{X{2cm}}}
        \pyc{print_matrix(V["K_i"][2][0], 1, 1e-3)}
      \end{array}
      \right]
    \cdot 10^3 \,\mathrm{SI}\text,
    \\
    \siplaces{4}
    \sifix{}
    \rmat K_2^\a = \left[
      \begin{array}{*{8}{X{2cm}}}
        \pyc{print_matrix(V["K_i"][2][1], 1, 1e-3)}
      \end{array}
      \right]
    \cdot 10^3 \,\mathrm{SI}\text.
  \end{align}
\end{myframe}

A globális merevségi mátrix meghatározásához írjuk fel az egyes elemekhez
tartozó szabadsági fokokat mátrixosan:
\begin{myframe}
  \def\arraystretch{1.15}
  \begin{equation}
    \rmat{DOF}^\a =
    \py{latex(sp.Matrix(V["DOF"][2]), mat_str="array").replace("cccc", "*{4}{X{7mm}}")}
    \text.
  \end{equation}
\end{myframe}

A globális merevségi mátrix összeállításakor figyelnünk kell arra, hogy az adott
elem merevségi mátrixának megfelelő elemeit a hozzá tartozó szabadsági fokhoz
tartozó helyhez rendeljük hozzá. Ezt a \ref{fig:K-construction-a}. ábra
szemlélteti.
\begin{figure}[ht]
  \centering
  \includestandalone{K-construction-2}
  \caption{A globális merevségi mátrix szemléletes felépítése}
  \label{fig:K-construction-a}
\end{figure}

A globális merevségi mátrix tehát az alábbi alakot veszi fel:
\begin{myframe}
  \begin{equation}
    \siplaces{4}
    \sifix{}
    \rmat K^\a = \left[
      \scalebox{.667}{$
          \begin{array}{*{6}{X{2cm}}}
            \pyc{print_matrix(V["K"][2], 1, 1e-3)}
          \end{array}
        $}
      \right]
    \cdot 10^3 \,\mathrm{SI}\text.
  \end{equation}
\end{myframe}

\subsection{A globális tömegmátrix}

Ebben a feladatrészben is először az egyes elemekhez tartozó tömegmátrixokat
fogjuk meghatározni, amely paraméteresen az alábbi alakot veszi fel:
\begin{myframe}
  \begin{equation}
    \rmat M_i
    = \frac{\rho_i A_i L_i}{420}
    \begin{bmatrix}
      156    & 22L_i     & 54     & -13L_i    \\
      22L_i  & 4{L_i}^2  & 13L_i  & -3{L_i}^2 \\
      54     & 13L_i     & 156    & -22L_i    \\
      -13L_i & -3{L_i}^2 & -22L_i & 4{L_i}^2  \\
    \end{bmatrix}
    \text.
    \label{eq:Mi-base}
  \end{equation}
\end{myframe}

Az elemi konzisztens tömegmátrixok numerikusan:
\begin{myframe}
  \begin{align}
    \siplaces{4}
    \sifix{}
    \rmat M_1^\a = \left[
      \begin{array}{*{8}{X{2cm}}}
        \pyc{print_matrix(V["M_i"][2][0], 1e-6, 1)}
      \end{array}
      \right]
    \,\mathrm{SI} \text,
    \\
    \siplaces{4}
    \sifix{}
    \rmat M_2^\a = \left[
      \begin{array}{*{8}{X{2cm}}}
        \pyc{print_matrix(V["M_i"][2][1], 1e-6, 1)}
      \end{array}
      \right]
    \,\mathrm{SI} \text.
  \end{align}
\end{myframe}

A globális tömegmátrix összeállítása az előzőekhez hasonlóan a
\ref{fig:K-construction-a}. ábra alapján:
\begin{myframe}
  \begin{equation}
    \siplaces{4}
    \sifix{}
    \rmat M_{\phantom{2}}^\a = \left[
      \scalebox{.667}{$
          \begin{array}{*{6}{X{2cm}}}
            \pyc{print_matrix(V["M"]["a"], 1e-6, 1)}
          \end{array}
        $}
      \right]
    \,\mathrm{SI}\text.
  \end{equation}
\end{myframe}

\subsection{A frekvencia-egyenlet}

Írjuk fel a frekvencia-egyenletet, ahol $\alpha$ a saját-körfrekvenciát jelöli:
\begin{myframe}
  \begin{equation}
    \det \left(
    {\rmat K^\a} - \alpha^2 {\rmat M^\a}
    \right) = 0
    \text.
  \end{equation}
\end{myframe}

Fontos, hogy az egyenlet csak abban az esetben oldható meg, ha a benne szereplő
mátrixok regulárisak. Mivel ez nem teljesül, ezért az egyenletet kondenzálnunk
kell. Írjuk fel a kényszerekből adódó peremfeltételeket. A görgős támasz az
$y$ irányú elmozdulást, a befogás pedig az elmozdulást és az elfordulást is
gátolja, vagyis:
\begin{myframe}
  \begin{equation}
    v_\py{P["not_free"][22]["pv"]} =
    v_\py{P["not_free"][22]["wv"]} =
    \varphi_\py{P["not_free"][22]["wphi"]} =
    0
    \text.
  \end{equation}
\end{myframe}

A kondenzált egyenletet úgy kapjuk meg, hogy a globális tömeg- és merevségi
mátrixból töröljük az elmozdulásban, vagy elfordulásban gátolt szabadsági
fokokhoz tartozó oszlopokat és sorokat:
\begin{myframe}
  \begin{alignat}{9}
    \widehat{\rmat M}^\a & =
    \siplaces{4}
    \sifix{}
    \left[\begin{array}{*{6}{X{2cm}}}
              \pyc{print_matrix(V["M_kond"]["a"], 1e-6, 1)}
            \end{array}\right]
    \,\mathrm{SI}\text,
    \\
    \widehat{\rmat K}^\a & =
    \siplaces{4}
    \sifix{}
    \left[\begin{array}{*{6}{X{2cm}}}
              \pyc{print_matrix(V["K_kond"][2], 1, 1e-3)}
            \end{array}\right]
    \cdot 10^3 \,\mathrm{SI}\text.
  \end{alignat}
\end{myframe}

A kondenzált frekvencia-egyenlet:
\begin{myframe}
  \begin{equation}
    \det \left(
    \widehat{\rmat K}^\a - \alpha^2 \widehat{\rmat M}^\a
    \right) = 0
    \text.
  \end{equation}
\end{myframe}

Az egyenletrendszer legkisebb három pozitív gyöke a szerkezet első három
saját-körfrekvenciája. A sajátfrekvenciák pedig a körfrekvenciák $2\pi$-ed
részei:
\begin{myframe}
  \begin{alignat}{9}
    \alpha_1^\a & = \pyc{prin_TeX(V["alpha"]["a"][0], "", "4")} \,/\, \mathrm{s} \text,
    \qquad      & \rightarrow \qquad
    f_1^\a      & = \pyc{prin_TeX(V["f"]["a"][0], "Hz", "4")}
    \text,                                                                              \\
    \alpha_2^\a & = \pyc{prin_TeX(V["alpha"]["a"][1], "", "4")} \,/\, \mathrm{s} \text,
    \qquad      & \rightarrow \qquad
    f_2^\a      & = \pyc{prin_TeX(V["f"]["a"][1], "Hz", "4")}
    \text,                                                                              \\
    \alpha_3^\a & = \pyc{prin_TeX(V["alpha"]["a"][2], "", "4")} \,/\, \mathrm{s} \text,
    \qquad      & \rightarrow \qquad
    f_3^\a      & = \pyc{prin_TeX(V["f"]["a"][2], "Hz", "4")}
    \text.
  \end{alignat}
\end{myframe}
