\section{3 elemes modell a koncentrált tömeg figyelembe vételével, konzisztens tömegmátrixszal}

\subsection{Általános adatok}

A csomópontok, az elemek, illetve ezek kapcsolata az előző feladatban
tárgyaltakkal megegyezik.

\subsection{A globális merevségi mátrix}

A globális merevségi mátrix alakja is megegyezik az előző feladatrészben
ismertetettekkel, vagyis:
\begin{myframe}
  \begin{equation}
    \rmat K^\c = \rmat K^\b
    \text.
  \end{equation}
\end{myframe}

\subsection{A globális tömegmátrix}

A globális tömegmátrix az $m_0$ koncentrált tömeg figyelembe vétele miatt
az alábbi alakot veszi fel:
\begin{myframe}
  \begin{equation}
    \rmat M^\c = \rmat M_{\text{elemek}}^\c + \rmat M_{m_0}^\c
    \text,
  \end{equation}
\end{myframe}
ahol az $\rmat M_\text{elemek}^\c$ mátrix értéke $\rmat M^\b$-vel egyezik meg,
$\rmat M_{m_0}^\c$ pedig az alábbi alakot veszi fel:
\begin{myframe}
  \begin{equation}
    \rmat M_{m_0}^\c = \left[
    \scalebox{.667}{$
      \py{my_latex(V["M_0_param"], mat_delim="", mat_str="array").replace("cc", "X{6mm}X{6mm}")}
    $}
    \right]
    \text.
  \end{equation}
\end{myframe}
A globális tömegmátrix tehát numerikusan:
\begin{myframe}
  \begin{equation}
    \siplaces{4}
    \sifix{}
    \rmat M^\c = \left[
      \scalebox{.667}{$
          \begin{array}{*{8}{X{2cm}}}
            \pyc{print_matrix(V["M"]["c"], 1e-6, 1)}
          \end{array}
        $}
      \right]
    \,\mathrm{SI}\text.
  \end{equation}
\end{myframe}

\subsection{A frekvencia-egyenlet}

A peremfeltételek megegyeznek a (\ref{eq:cond-3})-os egyenletben
szereplő feltételekkel. Ezek alapján a kondenzált mátrixok:
\begin{myframe}
  \begin{alignat}{9}
    \widehat{\rmat M}^\c & =
    \siplaces{4}
    \sifix{}
    \left[
      \scalebox{.833}{$
          \begin{array}{*{6}{X{2cm}}}
            \pyc{print_matrix(V["M_kond"]["c"], 1e-6, 1)}
          \end{array}
        $}
      \right]
    \,\mathrm{SI}\text,
    \\
    \widehat{\rmat K}^\c & =
    \siplaces{4}
    \sifix{}
    \left[
      \scalebox{.833}{$
          \begin{array}{*{6}{X{2cm}}}
            \pyc{print_matrix(V["K_kond"][3], 1, 1e-3)}
          \end{array}
        $}
      \right]
    \cdot 10^3 \,\mathrm{SI}\text.
  \end{alignat}
\end{myframe}

A kondenzált frekvencia-egyenlet:
\begin{myframe}
  \begin{equation}
    \det \left(
    \widehat{\rmat K}^\c - \alpha^2 \widehat{\rmat M}^\c
    \right) = 0
    \text.
  \end{equation}
\end{myframe}

A szerkezet első három saját-körfrekvenciája, és sajátfrekvenciája:
\begin{myframe}
  \begin{alignat}{9}
    \alpha_1^\c & = \pyc{prin_TeX(V["alpha"]["c"][0], "", "4")} \,/\, \mathrm{s}
    \qquad      & \rightarrow \qquad
    f_1^\c      & = \pyc{prin_TeX(V["f"]["c"][0], "Hz", "4")}
    \text,                                                                       \\
    \alpha_2^\c & = \pyc{prin_TeX(V["alpha"]["c"][1], "", "4")} \,/\, \mathrm{s}
    \qquad      & \rightarrow \qquad
    f_2^\c      & = \pyc{prin_TeX(V["f"]["c"][1], "Hz", "4")}
    \text,                                                                       \\
    \alpha_3^\c & = \pyc{prin_TeX(V["alpha"]["c"][2], "", "4")} \,/\, \mathrm{s}
    \qquad      & \rightarrow \qquad
    f_3^\c      & = \pyc{prin_TeX(V["f"]["c"][2], "Hz", "4")}
    \text.
  \end{alignat}
\end{myframe}
