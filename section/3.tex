\section{3 elemes modell a koncentrált tömeg elhanyagolásával}

\subsection{Általános adatok}

Az előző feladattal ellentétben a mostani végeselemes modellünk 4 csomópontot
és három gerenda-elemet tartalmaz. A csomópontok koordinátáit az
\ref{fig:dot-numbering-b}. táblázat tartalmazza.
\begin{table}[H]
  \caption{A csomópontok koordinátái}
  \centering
  \begin{tabular}{|c *{2}{|| X{15mm} | X{15mm} }|}
    \hline
    Csp. & x     & y & x \, [\mathrm{mm}] & y \, [\mathrm{mm}] \\ \hline\hline
    $1$  & 0     & 0 & 0                  & 0                  \\ \hline
    $2$  & b / 2 & 0 & \sipv{hb}{}        & 0                  \\ \hline
    $3$  & b     & 0 & \sipv{b}{}         & 0                  \\ \hline
    $4$  & b + a & 0 & \sipv{c}{}         & 0                  \\ \hline
  \end{tabular}
  \label{fig:dot-numbering-b}
\end{table}

Az egyes elemekhez tartozó csomópontokat, valamint a gerendák hosszát a
\ref{fig:beam-numbering-b}. táblázat foglalja össze.
\begin{table}[H]
  \caption{Elem – csomópont összerendelések}
  \centering
  \begin{tabular}{|c || c | c | c|}
    \hline
    Elem. & 1. csp & 2. csp & $L$   \\ \hline \hline
    $1$   & $1$    & $2$    & $b/2$ \\ \hline
    $2$   & $2$    & $3$    & $b/2$ \\ \hline
    $3$   & $3$    & $4$    & $a$   \\ \hline
  \end{tabular}
  \label{fig:beam-numbering-b}
\end{table}

\subsection{A globális merevségi mátrix}

Az elemi merevségi mátrixok az (\ref{eq:Ki-base})-es egyenlet alapján numerikusan:
\begin{myframe}
  \begin{align}
    \siplaces{4}
    \sifix{}
    \rmat K_1^\b = \rmat K_2^\b = \left[
      \begin{array}{*{8}{X{2cm}}}
        \pyc{print_matrix(V["K_i"][3][0], 1, 1e-3)}
      \end{array}
      \right]
    \cdot 10^3 \,\mathrm{SI}\text,
    \\
    \siplaces{4}
    \sifix{}
    \rmat K_3^\b = \left[
      \begin{array}{*{8}{X{2cm}}}
        \pyc{print_matrix(V["K_i"][3][2], 1, 1e-3)}
      \end{array}
      \right]
    \cdot 10^3 \,\mathrm{SI}\text.
  \end{align}
\end{myframe}

A globális merevségi mátrix meghatározásához írjuk fel az egyes elemekhez
tartozó szabadsági fokokat mátrixosan:
\begin{myframe}
  \def\arraystretch{1.05}
  \begin{equation}
    \rmat{DOF}^\b =
    \py{latex(sp.Matrix(V["DOF"][3]), mat_str="array").replace("cccc", "*{4}{X{7mm}}")}
    \text.
  \end{equation}
\end{myframe}

A globális merevségi mátrix összeállításakor figyelnünk kell arra, hogy az adott
elem merevségi mátrixának megfelelő elemeit a hozzá tartozó szabadsági fokhoz
tartozó helyhez rendeljük hozzá. Ezt a \ref{fig:K-construction-b}. ábra
szemlélteti.
\begin{figure}[ht]
  \centering
  \includestandalone{K-construction-3}
  \caption{A globális merevségi mátrix szemléletes felépítése}
  \label{fig:K-construction-b}
\end{figure}

A globális merevségi mátrix tehát az alábbi alakot veszi fel:
\begin{myframe}
  \begin{equation}
    \siplaces{4}
    \sifix{}
    \rmat K^\b = \left[
      \scalebox{.667}{$
          \begin{array}{*{8}{X{1.85cm}}}
            \pyc{print_matrix(V["K"][3], 1, 1e-3)}
          \end{array}
        $}
      \right]
    \cdot 10^3 \,\mathrm{SI}\text.
  \end{equation}
\end{myframe}

\subsection{A globális tömegmátrix}

Az elemi konzisztens tömegmátrixok a (\ref{eq:Mi-base})-os egyenlet alapján
numerikusan:
\begin{myframe}
  \begin{align}
    \siplaces{4}
    \sifix{}
    \rmat M_1^\b = \rmat M_2^\b = \left[
      \begin{array}{*{8}{X{2cm}}}
        \pyc{print_matrix(V["M_i"][3][0], 1e-6, 1)}
      \end{array}
      \right]
    \,\mathrm{SI} \text,
    \\
    \siplaces{4}
    \sifix{}
    \rmat M_3^\b = \left[
      \begin{array}{*{8}{X{2cm}}}
        \pyc{print_matrix(V["M_i"][3][2], 1e-6, 1)}
      \end{array}
      \right]
    \,\mathrm{SI} \text.
  \end{align}
\end{myframe}

A globális tömegmátrix összeállítása az előzőekhez hasonlóan a
\ref{fig:K-construction-b}. ábra alapján:
\begin{myframe}
  \begin{equation}
    \siplaces{4}
    \sifix{}
    \rmat M_{\phantom{2}}^\b = \left[
      \scalebox{.667}{$
          \begin{array}{*{8}{X{2cm}}}
            \pyc{print_matrix(V["M"]["b"], 1e-6, 1)}
          \end{array}
        $}
      \right]
    \,\mathrm{SI}\text.
  \end{equation}
\end{myframe}

\subsection{A frekvencia-egyenlet}

Írjuk fel a frekvencia-egyenletet, ahol $\alpha$ a saját-körfrekvenciát jelöli:
\begin{myframe}
  \begin{equation}
    \det \left(
    {\rmat K^\b} - \alpha^2 {\rmat M^\b}
    \right) = 0
    \text.
  \end{equation}
\end{myframe}

A peremfeltételek jelen esetben:
\begin{myframe}
  \begin{equation}
    v_\py{P["not_free"][33]["pv"]} =
    v_\py{P["not_free"][33]["wv"]} =
    \varphi_\py{P["not_free"][33]["wphi"]} =
    0
    \text.
    \label{eq:cond-3}
  \end{equation}
\end{myframe}

A kondenzált mátrixok:
\begin{myframe}
  \begin{alignat}{9}
    \widehat{\rmat M}^\b & =
    \siplaces{4}
    \sifix{}
    \left[
      \scalebox{.833}{$
          \begin{array}{*{6}{X{2cm}}}
            \pyc{print_matrix(V["M_kond"]["b"], 1e-6, 1)}
          \end{array}
        $}
      \right]
    \,\mathrm{SI}\text,
    \\
    \widehat{\rmat K}^\b & =
    \siplaces{4}
    \sifix{}
    \left[
      \scalebox{.833}{$
          \begin{array}{*{6}{X{2cm}}}
            \pyc{print_matrix(V["K_kond"][3], 1, 1e-3)}
          \end{array}
        $}
      \right]
    \cdot 10^3 \,\mathrm{SI}\text.
  \end{alignat}
\end{myframe}

A kondenzált frekvencia-egyenlet:
\begin{myframe}
  \begin{equation}
    \det \left(
    \widehat{\rmat K}^\b - \alpha^2 \widehat{\rmat M}^\b
    \right) = 0
    \text.
  \end{equation}
\end{myframe}

A szerkezet első három saját-körfrekvenciája, és sajátfrekvenciája:
\begin{myframe}
  \begin{alignat}{9}
    \alpha_1^\b & = \pyc{prin_TeX(V["alpha"]["b"][0], "", "4")} \,/\, \mathrm{s}
    \qquad      & \rightarrow \qquad
    f_1^\b      & = \pyc{prin_TeX(V["f"]["b"][0], "Hz", "4")}
    \text,                                                                       \\
    \alpha_2^\b & = \pyc{prin_TeX(V["alpha"]["b"][1], "", "4")} \,/\, \mathrm{s}
    \qquad      & \rightarrow \qquad
    f_2^\b      & = \pyc{prin_TeX(V["f"]["b"][1], "Hz", "4")}
    \text,                                                                       \\
    \alpha_3^\b & = \pyc{prin_TeX(V["alpha"]["b"][2], "", "4")} \,/\, \mathrm{s}
    \qquad      & \rightarrow \qquad
    f_3^\b      & = \pyc{prin_TeX(V["f"]["b"][2], "Hz", "4")}
    \text.
  \end{alignat}
\end{myframe}
